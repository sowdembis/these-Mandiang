\documentclass[a4paper,11pt]{article}
\usepackage{tikz}
\usetikzlibrary{decorations.markings}
%\usepackage{tkz-base,tkz-fct}
%\usetikzlibrary{automata, arrows}
\usepackage{times}
\usepackage{graphicx}
\usepackage{float}
\usepackage{pgfplots}
\usepackage{amssymb}
\usepackage{amstext}
\usepackage{authblk}
\usepackage{multicol}
\usepackage{paralist}
\usepackage{url}
\usepackage{bm}
\usepackage{thmtools,thm-restate}
\usepackage[centertags]{amsmath}
\usepackage{algorithm,algorithmic}
\usepackage{hyperref}
\usepackage{amsthm}
%\usepackage[centertags]{amsmath}
\newtheorem{theorem}{Theorem}[section]
\newtheorem{proposition}[theorem]{Proposition}
\newtheorem{remark}[theorem]{Remark}
%\newtheorem{proof}[theorem]{Proof}
\newtheorem{definition}[theorem]{Definition}
\newtheorem{lemma}[theorem]{Lemma}

\newtheorem{corollary}[theorem]{Corollary}


\begin{document}

\title{Ring Signature Based on Generalized ElGamal Signature Scheme}
\author{\scriptsize{Demba Sow$^{1}$ and Mohamed MANDIANG$^{2}$}}
\affil{D\'epartement de Math\'ematiques et Informatique, FST, UCAD$^{1}$\\
{\texttt{demba1.sow@ucad.edu.sn$^{1}$}}
\\ Section Math\'ematiques Appliqu\'ees, UFR SAT, UGB$^{2}$\\
 {\texttt{mohamed.sn@gmail.com$^{2}$}}}

\maketitle


\tableofcontents

    \begin{abstract}

    \end{abstract}

\section{Introduction}

\paragraph{Contributions:} Our main aim is ...

\begin{itemize}
 \item
 \item
 \item
\end{itemize}

\paragraph{Related works:} Ring Signatures schemes ...

In~\cite{ringsignOrigine}

In~\cite{Elgamal}, ...


In~\cite{ringSignElG}, ...


In~\cite{sow}, ...



\paragraph{Outline:} This paper is organized as follows:
\begin{itemize}
 \item In Section~\ref{sec:two}, ...
 \item In Section~\ref{sec:three}, ...
 \item In Section~\ref{sec:four}, ...
\end{itemize}


\section{Preliminaries}\label{sec:two}

\subsection{Ring Signature}\label{sec:two:1}

\subsection{Security Notions}\label{sec:two:2}

\subsection{Generalized ElGamal Signature}\label{sec:two:2}
We present the key generation mechanism, the signature and verification
algorithms~\cite{sow}, which can be view as a slight modification of
ElGamal's scheme~\cite{Elgamal}.

\paragraph{Key generation algorithm.}
    To create a public/private key, we do the following:
\begin{itemize}
    \item Select a cyclic group $G$ with sufficiently large order $d$ such that $G=\langle g \rangle$.

    \item Select two random integers $r$ and $k$ sufficiently large
      such that $ 2 < k < d$ and $r$ of size half the size of $d$ and compute $kd$.
    \item Compute with Euclidean division algorithm, the pair $(s,t)$
      such that $kd= rs+t$ where $t = kd \mod s$.

    \item  Compute $\gamma = g^{s} $ and $\delta = g^{t} $ in $G$; Note that $\gamma \neq 1$ and $\delta \neq 1$.
\end{itemize}
Then public key is $((\gamma, \delta), G)$ and the private key is $( r, G)$.

\paragraph{Signature algorithm.}
    To sign a message $m$ with the private key $(r, G)$, we do the following:
\begin{enumerate}
    \item Choose a random integer $2<\beta< n=\#G$ such that $\gcd( \beta, n)=1$, $ \beta$ and $d - \beta$  must be sufficiently large;

    \item Compute $R=(g^{s})^{r  \beta}$ in $G$, $H(h(R))$, $(r  \beta)^{-1} \mod n$ and $ s^{-1}t \mod n$;

    \item Compute $S= (r  \beta)^{-1} \{H(h(m)) - s^{-1}tH(h(R))\} \mod n$; if $S=0$ or $S=1$, return to stage 2;

    \item Output $(R, S)$ as the signature of $m$.

\end{enumerate}

\paragraph{Verification algorithm.}
 To verify a signture $(R, S)$ of a message $m$, we do the following:
 
 \begin{enumerate}
    \item Take $(\gamma, \delta, G)=((g^{s},g^{t}), G)$ the public key and $(R, S)$ the signature of $m$;

    \item Compute $V_{1}= (g^{t})^{H(h(R))}R^{S} $  and  $V_{2}= (g^{s})^{H(h(m))}$  in $G$;

    \item The signature is valid if $V_{1}=V_{2}$;
 
\end{enumerate}

\section{Ring Signature Based on ElGamal Signature}\label{sec:three}

\paragraph{Key Generation}
    \begin{itemize}
      \item 
    \end{itemize}

\paragraph{Signature Algorithm}

    \begin{itemize}
      \item 
    \end{itemize}

\paragraph{Verification Algorithm}
    
    \begin{itemize}
      \item 
    \end{itemize}


\section{Ring Signature Based on Generalized ElGamal Signature}\label{sec:four}

\paragraph{Key Generation}

\begin{enumerate}
    \item 
\end{enumerate}

\paragraph{Signature Algorithm}
     

\begin{enumerate}
    \item 
\end{enumerate}

\paragraph{Verification Algorithm}
    
\begin{enumerate}
    \item 
\end{enumerate}

\paragraph{Correctness}
    \begin{itemize}
      \item 

      \item 
    \end{itemize}

\section*{\textbf{Conclusion}}

\bibliographystyle{alpha}
\bibliography{biblio}

\end{document}

